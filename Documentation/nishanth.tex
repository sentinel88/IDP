\documentclass{article}

\usepackage{amssymb}
\usepackage{verbatim}
\setcounter{tocdepth}{3}
\usepackage{graphicx}
\usepackage{listings}
\usepackage{subfigure}
\usepackage{wrapfig}
\usepackage[hyperindex=false,colorlinks=false]{hyperref}
\usepackage{fullpage}

\begin{document}

\title{Implementation of a Metaheuristic for the Discrete Network Design Problem}
\author{Nishanth Nagendra}

\maketitle

\begin{abstract}
The field of transportation planning involves a large class of problems that are characterized by multiple levels of decision making. Examples include selection of links for capacity improvements(network design), toll setting, traffic signal setting etc. In each of these problems, government or industry officials make one set of decisions to improve the network performance and at another level users make choices with regard to route, origin-destination etc. The discrete network design problem(DNDP) is one such problem which deals with the selection of link additions to an existing road network, with given demand from each origin to destination. The objective is to make an optimal investment decision to minimize the total travel cost in the network, while accounting for the route choice behavior of network users. This optimization problem is recognized to be NP hard because it is computationally difficult due to its non convexity owing to the bilevel nature and non-linear objective functions in most of the real cases. Finding exact or globally optimum solutions for such problems is very difficult. In mathematical optimization, a metaheuristic is a general high level procedure that can be quickly applied to different kinds of problems to provide a sufficiently good solution especially when there is limited computation capacity. These techniques sample a set of solutions which is too large to be completely sampled. Genetic algorithm is one such kind of a metaheuristic which tries to imitate the evolution of population by starting with a random set of candidate solutions that is evolved towards better solutions. The aim of the project is to implement this metaheuristic on the DNDP and to experiment with small to large size datasets(networks) widely mentioned in the literature to derive empirical results and to conclude on the effectiveness, solvability and quality of solutions obtained.
\end{abstract}


\section{Introduction}
The network design problem(NDP) is concerned with the modification of a transport infrastructure by adding new links or improving existing ones. In most applications, one is interested in selecting from among a relatively small set of improvements to an existing network rather than designing an entirely new network from scratch. Network design problems arise in many transportation modes: urban mass transit, highway, rail etc. Most applications have been in highway improvement, however. The discrete network design problem(DNDP) deals with the selection of link additions to an existing road network, with given demand from each origin to destination. The objective is to make an optimal investment decision in order to minimize the total travel cost in the network, while accounting for the route choice behavior of network users. Another form of this problem is the continuous network design problem(CNDP) which deals with the optimal capacity expansion of existing links.\par
\noindent
\\These combinatorial problems are recognized as NP Hard due to the considerable amount of computational difficulties faced in trying to solve them because of their non-convex nature and due to the form of a bi-level mixed integer program with a large number of 0-1 variables. This Inter disciplinary project deals with only the discrete form of the NDP and considers the following kind of problem which often arises in transportation planning: minimize total travel expenditure subject to a budget constraint on the total construction cost of new links.
 
\subsection{Problem Formulation}
Bilevel problems are mathematical programming problems consisting of a special kind of optimization where one problem is embedded within another. The outer optimization task is commonly referred to as the upper level(or the leader) optimization task with a nested optimization problem(the lower one - also called follower) in the constraints. First, the leader has to decide over a subset of the decision variables which effects the objective value of the follower. Afterwards, the follower has to decide over the other subset of decision variables, which affects the objective value of the leader. The problem we are considering to solve here is a bilevel linear program with binary leader variables and continuous follower variables. An application of bilevel programming is network design problems where the objective of the network designer is to reduce congestion in the network and the objective of the follower is to find the fastest way from origin to destination. Existing methods for solving this problem can be roughly divided into the following categories:\par
\begin{itemize}
\item Methods based on vertex enumeration
\item Methods based on kuhn tucker conditions
\item Fuzzy approach
\item Methods based on meta heuristics
\end{itemize}

\subsection{Meta Heuristics}
Modern heuristic techniques, also called metaheuristics are a family of procedures which benefit from some sort of intelligence in their search for finding the solution to a problem. It is a higher level procedure designed to find, generate, or select a heuristic(partial search algorithm) that may provide a sufficiently good solution to an optimization problem, especially with incomplete or imperfect information or limited computation capacity. They may not provide optimal solution but they provide a sufficiently good solution rapidly and effectively. Simulated annealing, genetic algorithm, tabu search, neural network, ant system are some examples of such meta heuristics.\par
\noindent
\\This project implements the genetic algorithm on DNDP. Genetic algorithm (GA) is a search heuristic that mimics the process of natural selection. This meta heuristic is used to generate useful solutions to optimization and search problems. Genetic algorithms belong to the larger class of evolutionary algorithms (EA), which generate solutions to optimization problems using techniques inspired by natural evolution, such as inheritance, mutation, selection, and crossover. A generic pseudo code of GA is given below:
\begin{lstlisting}[mathescape]
	Initialize population P
	Repeat
	$P'$ = { }
	for i = 1 to n
	   x := Selection of individual of P
	   y := Selection of individual of P
	   child := Crossover(x, y)
	   if random(0,1) $\leq$ (probability of mutation) then
	   	child := Mutation(child)
	   $P'$ := $P'$ $\cup$ child
	end for
	P := $P'$
\end{lstlisting}
\begin{itemize}
\item Stopping criteria: Time limit, iteration limit, objective function value reached.
\end{itemize}
The genetic algorithm will be used to solve the upper level problem whereas the nested problem is solved using BCL component of FICO Xpress software. For every candidate solution in the upper level problem the nested problem will be solved using BCL. This iterative process is continued to evolve towards stronger solutions for the leader by the use of genetic algorithm operations like crossover, mutation etc till the stopping criteria is satisfied.

\subsection{InterDisciplinary Project}
The implementation of this project requires knowledge of C/C++ programming, linear programming, usage of the C++ BCL(Builder component library) provided by a commercially well known optimization suite called FICO Xpress. The application of these technologies is for a well known real world problem like DNDP observed in transportation science resulting in this project to be of Inter-Disciplinary nature. As a part of the preparation for implementing this project, a good amount of knowledge on metaheuristics and DNDP was acquired from the course "Transportation logistics" offered by the chair of logistics and supply chain management. A rough timeline from when the project started till its anticipated completion time is given below.
\begin{itemize}
\item Dec 2014 - Start of IDP with initial literature survey but main focus on the course.
\item Early Feb 2015 to Mid April 2015 - Further literature reading, setup, understanding and practical usage of the FICO Xpress software, writing code using BCL component.
\item Mid of April 2015 to present - Implementation phase with good amount of testing. Further testing will follow on larger datasets with additional code to provide extensions. This will be followed up by documentation and presentation resulting in an expected completion time of early Sep 2015.
\end{itemize}

%\begin{figure}[h]
%\centering
%\includegraphics[width=0.65\textwidth, clip, trim=0mm 60mm 0mm 0mm]{data/architecture.pdf}
%\vspace{-0.15in}
%\caption{Invasive resource management architecture}
%\label{fig:arch}
%\end{figure}

\end{document}
